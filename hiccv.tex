\documentclass{article}
\usepackage[latin1]{inputenc}               % Scandivanian characters
\usepackage{fancyhdr}                       % headers on every page
\usepackage{charter}
\usepackage[T1]{fontenc}
\usepackage{hiccurrvita}                    % currvita modified to handle
                                            % multiple biographies
\usepackage[unitcntnoreset]{bibtopic}       % multiple biographies
\bibliographystyle{ieeetr}                  % numbered citations
\usepackage{tocloft}                        % tabs with dots
\usepackage{graphicx}                       % inclusion of graphics
\usepackage[colorlinks=true,plainpages=false]{hyperref}
\usepackage{microtype}
\setlength{\topmargin} {-0.30in}
\setlength{\textwidth} {6.85in}
\setlength{\textheight}{8.50in}
\evensidemargin -.1in
\oddsidemargin  -.1in

% \setlength{\itemsep}{0ex}

\newenvironment{sublist}{%
  \begin{list}{}{%
      \setlength{\itemsep}{0em}\setlength{\parsep}{0em}%
      \setlength{\topsep}{0em}\setlength{\parskip}{0em}%
    }%
}%
{ \end{list} }

\newenvironment{subbulletlist}{%
  \begin{list}{\labelitemii}{%
      \setlength{\topsep}{\itemsep}\setlength{\parskip}{\parsep}%
    }%
}%
{ \end{list} }

% -- headers --------------------------------------------------
\rhead{Henrik Iskov Christensen}
\lhead{Curriculum Vitae}
\rfoot{\today}
\pagestyle{fancy}
\title{CV for Henrik Iskov Christensen}


% -------------------------------------------------------------
\begin{document}
\pagenumbering{arabic}
\begin{tabular}{cl}
  \begin{minipage}{4cm}
    \includegraphics[width=3.5cm]{hic-face-2019}
  \end{minipage}&
  \begin{minipage}{10cm}
    {\Large
      \begin{center}
        Curriculum Vitae\\
        for\\
        Dr.\ Henrik Iskov Christensen
      \end{center}
    }
  \end{minipage}
\end{tabular}
\vspace{5mm}

% -------------------------------------------------------------

\begin{cv}

  \noindent{\bf \large I. PERSONAL DATA}

  \noindent
  \begin{cvlist}{~}
  \item[Name:]\ \\
    Henrik Iskov Christensen,\\
    Born: Frederikshavn, Denmark\\
  \item[Address:]\ \\
    3275 5th Avenue, Unit 501\\
    San Diego, CA 92103\\
    Phone: +1 858 260 0570\\
    Email: \url{henrik@hichristensen.com}\\
  \item[Affiliation:]\ \\
    Contextual Robotics Institute\\
    Jacobs School of Engineering\\
    Univ.~of California --- San Diego\\
    9500 Gilman Drive\\
    La Jolla, CA 92093--0436\\
    Cell: +1 404 889 2500\\
    Email: \url{hichristensen@ucsd.edu}\\
  \item[Citizenship:] USA (Naturalized Dane)
  \end{cvlist}

  \begin{cvlist}{Professional interests:}
  \item A {\em systems}\/ oriented approach to Machine Perception,
    Robotics, and Artificial Intelligence for Autonomy
  \end{cvlist}

% -------------------------------------------------------------

\noindent{\bf \large II.\ EDUCATIONAL BACKGROUND}

  \begin{cvlist}{~}
  \item[1989] Ph.D., Faculty of Technical Sciences, Aalborg
    University, DK
    \begin{sublist}
    \item Major subjects: Motion Analysis, Multi Scale Image
      Representation of Space and Time, and Concurrent computing.
    \item Dissertation: ``Aspects of Real Time Image Sequence
      Analysis''
    \item Supervisor: Prof.\ Erik Granum.
    \end{sublist}
  \item[1987] M.Sc. EE (Summa Cum Laude), Institute of Electronic
    Systems, Aalborg University, DK
    \begin{sublist}
    \item Major subjects: Process Control and Image Analysis
    \item Thesis: ``Monitoring Moving Objects in Real-Time''
    \end{sublist}
  \item[1981] Mechanical Design, Cert.
    of Apprenticeship (with honors), Frederikshavn Technical
    College, Denmark.
\end{cvlist}

% -------------------------------------------------------------

\noindent{\bf \large III.\ EMPLOYMENT}

\begin{cvlist}{Professional Experience:}


\item[Feb 2019 --] Co-founder of Robust.AI, Palo Alto, CA

\item[Jul 2017 --] Qualcomm Chancellor's Chair in Robot Systems, UC San Diego.

\item[Aug 2016 --] Director Institute for Contextual Robotics, UC San
  Diego. The Institute studies robot systems in the context of its
  actual use-cases and real-time perceptual information to provide
  holistic and robust solutions. The Institute involves four
  departments across Engineering and Social Sciences.

\item[Aug 2016 --] Distinguished Professor of Computer Science, Dept.\ of
  Computer Science and Engineering, UC San Diego, California, USA.\@

\item[Oct 2016 --Jul 2018] Member Board of Director, Blue Ocean Robotics Ltd,
  Odense, Denmark.

\item[Feb 2014 -- 2016] Advisor to the Manufacturing Academy of
  Denmark (MADE) with particular emphasis on strategy and impact. MADE
  is a joint venture between Danish Production Companies, The Council
  for Strategic Research and 4 Danish Universities.

\item[Jan 2014 -- Sep 2016] Board of Directors, ``Universal Robots
  Inc''. New York, NY.\@ The US subsidiary of Universal Robots, Odense,
  DK

\item[Oct 2013 --Jul 2016] Founding Executive Director, ``Institute
  for Robotics and Intelligent Machines'', Georgia Institute of
  Technology -- IRIM --- a unit that involves more than 60 faculty and
  150 graduate students doing research, education and translation of
  robotics across manufacturing, services, healthcare and defense
  applications.

\item[Sep 2013--Apr 2023] Co-Founder of ``Robo-Global'' a benchmark index to
  track the global robotics and automation market. Chairman of AI Strategy and
  general robotics advisor. Robo-Stox, is an exchange traded fund (ETF) on NYSE
  with the ticker symbol ROBO.\@ A parallel fund was launched at London Stock
  Exchange 2014. The international company was renamed Robo-Global from
  Robo-Stox January 2016.\@ Robo-Global was acquired by VettaFi April 2023.

\item[June 2013--] Co-founder/Partner of  ``Christensen
  Consulting Group'', providing services to government agencies.

\item[May 2012--2014] Co-founder/Partner of the company The ``OR
  Standard'' --- a company that optimizes workflow for hospital
  operating suites. Initially, seeded through the GT Flashpoint
  Program.

\item[Aug 2009--2017] Adjunct Professor of Electrical and Computer
  Engineering, College of Engineering, Georgia Institute of
  Technology.

\item[Aug 2006--Oct 2013] Director Georgia Tech Center for Robotics and
  Intelligent Machines (RIM@GT), an interdepartmental research center
  involving College of Computing, College of Engineering, and GTRI.\@

\item[Feb 2006 --Aug 2016]  Professor of Computer Science/Kuka
  Chair of Robotics --- Georgia Institute of Technology, Atlanta, GA,
  USA.\ Part time during 2006 and full time from January 2007.

\item[July 1998--Dec 2006] Chaired Professor of Computer Science,
  Dept.\ of Numerical Analysis and Computer Science, Royal Institute
  of Technology, Stockholm, Sweden. On part-time leave during 2006

\item[Sept. 1996--July 2006] Scientific Director for ``Center for
  Autonomous Systems'', Kungliga Tek\-ni\-ska H\"og\-skolan,
  Stock\-holm.  Sponsored by the Swedish Strategic Research
  Foundation. Associated with the Computational Vision and Active
  Perception group, Dept.~for Numerical Analysis and Computer
  Science.

\item[Jan. 1996--July 1996] Visiting Professor of Computer and
  Information Science.  GRASP Laboratory, University of Pennsylvania.
  Doing research on non.linear dynamical systems for control of
  autonomous sensor-driven agents and intelligent control for
  multi-agent systems.

\item[Apr. 1992--Aug. 1998] Associate professor specializing in robot
  and computer vision, Faculty of Technical Sciences, Aalborg
  University.  Project manager for internationally funded research
  projects.

\item[June 1992--Sept. 1995] Local manager for the ESPRIT Basic
  Research project ``Vision as Process-II''.  Principal investigator
  wrt.\ control of perception.

\item[Jan. 1990--Dec. 1993] Chairman of the National Vision
  Programmers Workbench (VIPWOB) group, that developed generic
  application architectures for image analysis and computer vision.

\item[Oct. 1989--April 1992] Research Associate and project head
  at Laboratory of Image Analysis.  Project: ESPRIT Basic Research
  Action BR3038-VAP, ``Vision as Process''.  The project was a
  collaboration with five other European Universities.  Primary topic
  for AUC work was perceptual control for dynamic vision systems.

\item[July 1988--Jan. 1989] Pre-doctoral Fellow at the Advanced
  Computing and Integrated Sensor Systems Group, Oak Ridge National
  Laboratory, Tennessee, USA.\@ Participating in the research programme
  ``Robotics and Intelligent Systems Program'' (RISP). Primary topics
  were concurrent computer vision, sensor fusion for mobile robots,
  and multi resolution methods for dynamic scene analysis. Sponsored
  by the Danish Technical Research Council, The Danish Research
  Academy, the Foundation Vision North, and U.S. Department of Energy,
  under contract DE-AC05-840R214.

\item[July 1987--Sept. 1989] Research assistant (Ph.D. student) sponsored
  by project ``Computer Vision, --- methods for real-time image
  sequence analysis''. Supported by FTU grant no. 5.17.5.6.06 from the
  Danish Technical Research Council.

\item[1986--1987] Part time programmer \& Teaching Assistant,
  Image Analysis Group, Aalborg University, DK.\@

\item[1980] Trainee as ``Technical Assistant'', Department of
  Automation, B\&W/MAN Alpha Diesel A/S, Frederikshavn, DK.\@
\end{cvlist}

\begin{cvlist}{Consulting Experience:}
\item Industrial Consultant:
\begin{sublist}
  \item Member Board of Directors --- AlphaTrAI (2023--)
  \item Senior Advisor --- Calibrate Ventures (2023--)
  \item Advisor --- Spring Mountain Capital (2022--)
  \item Research Advisory Board --- AutoDesk (2021--2023)
  \item Dean's Advisory Board --- Fowley School of Engineering, Chapman University (2020--2024)
  \item OnRobot --- Advisor / Board Member (2018--2019)
  \item Covariant.AI --- Advisory Board (2018--2019)
  \item Ready Robotics --- Chair of Advisory Board (2017--2019)
  \item Tech Mahindra --- Manufacturing Advisory Board (2015--2018)
  \item General Electric --- Robotics Advisory Board (2015--2016)
  \item The Boeing Company --- Strategic Automation Roadmap (2015)
  \item DARPA DRC Impact Study \& Student Contest (2013--2015)
  \item CLSA --- Robotics \& Automation (2014--2016)
  \item Symbotic (2014)
  \item MAG-IAS Composites Board of Directors (2010--2013)
  \item C\&S WholeSale (2010--2011)
  \item Scientific Advisor to Evolution Robotics (1998--2012)
  \item Member of International Advisory Committee, Instituto Robotica e
    Systemas, Lisboa, Portugal, (1998--2008)
  \item iRobot (2004--2006)
  \item ABB (2001--2005)
  \item Totoya Motor Company (2000--2004)
  \item Member of the European Image Understanding Environment Design
    Committee, (1994--1997)
  \item Registered consultant for Apple Computers Inc. (DK, 1989--1995)
  \item Development of a system for automatic real time obstacle
    detection at rail road crossings (DSB, 1988)
  \end{sublist}
\end{cvlist}

% -------------------------------------------------------------

\noindent{\bf \large IV.\ TEACHING}

\begin{cvlist}{~}

\item[Ph.D supervision - Ongoing]\ \\
\begin{enumerate}
    \item ``Prediction for Autonomous Vehicles'', Jing-yan Liao, CS@UCSD (2028)
    \item ``Intent Recognition for Autonomous Systems'', Zihan Zhang, CS@UCSD (2028)
    \item ``Behavior Planning in Urban Environments'', Luobin Wang, CS@UCSD (2028)
    \item ``Adversarial Planning in Game Theory'', Rohan Patil, CS@UCSD (2028)
    \item ``Mapping for Autonomous Vehicles'', Seth Farrell, CS@UCSD (2028)
    \item ``Autonomy for First-Responder Support'', Julian Raheema, CS@UCSD (2028)
    \item ``Hybrid Fusion for Situation Awareness'', Henry Zhang, CS @  UCSD (2025)
    \item ``Robust Dexterous Manipulation'', Jiaming Hu, CS @ UCSD (2025)
    \item ``Knowledge Based Mapping'', Yiding Qiu, CS @ UCSD (2025)
    % \item ``Affordance based manipulation'', Andrea Frank, CS @ UCSD (2024)
  \end{enumerate}

\item[Ph.D supervision - Completed]\ \\
  \begin{enumerate}
    \item ``Achieving Flow'', Chris D'Ambrosia, CS@UCSD (2023)
    \item ``Object Based Mapping'', Anwesan Pal, CS @ UCSD (2023)
    \item ``Leveraging Contextual Knowledge Allows Service Robots to
          Efficiently Organize Household Objects in the Real World'', Akanimoh Adeleye, CS @ UCSD (2023)
    \item ``Autonomous Micro-Mobility'', David Paz, CS @ UCSD (2023)
    \item ``Robot learning through Reinforcement Learning, Teleoperation
        and Scene Reconstruction'', Quan Voung, CS @ UCSD (2022)
    \item ``Using Meta-Reasoning for Failure Detection and Recovery for
          Assembly Robots'', Priyam Parashar, CS @ UCSD (2021)
    \item ``Robotics Software Engineering'', Ruffin White, CS @ UCSD (2021)
    \item ``StereoFlow Camera'', Dominique Meyer, CS@UCSD, (2021)
          (Co-Advised w. Falko Kuester)
    \item ``Affordance based planning for manipulation'', Andrew Price, Robotics
          @ GT (2021 - Joint with S. Balakirsky)
    \item ``Multi-Robot Navigation and Mapping'', Carlos Nieto, ECE @ UCSD
          (2021)
    \item ``Robust Autonomy'', Shengye Wang, CS @ UCSD (2020)
    \item ``Object-based SLAM'', Siddharth Choudhary, Robotics @ GT (2017)
    \item ``Grasp Planning'', Ana Huaman, Robotics @ GT (2016)
    \item ``Planning in Constraint Space for Multi-body Manipulation Tasks'',
          Can Erdogan, Robotics @ GT (2016)
    \item ``Navigation Behavior Design and Representations For a People Aware
          Mobile Robot System'', Akansel Cosgun, Robotics @ GT (2016)
    \item ``Model Based SLAM'', Alexander Trevor, Robotics @ GT (2015)
    \item ``Autonomous environment manipulation to facilitate task completion'',
          Martin Levihn, Robotics @ GT (2015)
    \item ``Time-Optimal sampling-based motion planning for manipulators with
          acceleration limits'', Tobias Kunz, Robotics @ GT (2015)
    \item ``Multi-Modal Object Tracking'', Changhyun Choi, Robotics @ GT (Aug
          2014)
    \item ``Knowledge Transfer in Robot Manipulation Tasks'', Jake Huckaby,
          Robotics @ GT (Mar 2014)
    \item ``Trust and reputation in dynamic, heterogenous multi-agent teams'',
          Charles Pippin, CS @ GT (Oct 2013)
    \item ``Life-long Mapping and Exploration with a Mobile Robot'', John Rogers
          III, Robotics @ GT (2012)
    \item ``HRI for Domestic Robots'', Ja-Young Sung, HCC @ GT (2011)
          (Co-advisor w. Beki Grinter)
    \item ``High Performance Manipulation'', Christian Smith -- CS @ KTH (Dec.
          2009)
    \item ``Semantic SLAM'', Elin-Anna Topp CS @ KTH (Lic. Oct 2006, Oct 2008)
    \item ``Deployment of Field Robots in Hazardous Environments'', Carl
          Lundberg, CS @ KTH (Dec 2007)
    \item ``Evolutionary Learning for CyberRodents'', Stefan Elfwing. CS @ KTH
          (Nov 2007)
    \item ``Information Fusion'', Ronnie Johansson, CS @ KTH (Lic. - Dec. 2003,
          Ph.D.\ Apr. 2006)
    \item ``Large Scale SLAM'', John Folkesson, CS @ KTH (Oct 2005)
    \item ``Architectures for Autonomous Systems'', Anders Oreb{\"a}ck, CS @ KTH
          (Dec. 2004)
    \item ``Attention Systems'', Ola Ramstr\"om, CS @ KTH (Lic. Nov 2004)
    \item ``Learning in Behavior Based Systems'', Philipp Althaus, CS @ KTH
          (November 2003)
    \item ``Structure from Motion'', Marco Zucchelli, CS @ KTH (June 2002)
    \item ``A Framework for Integration of Processes'', Lars Petersson, CS @ KTH
          (Mar 2002)
    \item ``Sensor Fusion for Navigation'', Guido Zunino, CS @ KTH (Lic. Feb
          2002)
    \item ``Visual Servoing for Manipulation: Robustness and Integration
          Issues'', Danica Kragic, CS @ KTH (June 2001)
    \item ``Approaches to Mobile Robot Localisation in Indoor Environments'',
          Patric Jensfelt EE @ KTH (June 2001)
    \item ``Sonar Based World Modelling'', Olle Wijk. EE @ KTH (April 2001)
    \item ``Towards Human-Robot Interaction'', Kristian Simsarian, CS @ KTH
          (Mar. 2000)
    \item ``Architectures for Autonomous Mobile Robot Navigation'', Paolo
          Pirjanian, CE @ AUC (April 1998)
    \item ``Sensor Planning for Mobile Robot Navigation'', Steen Kristensen. EE
          @ AUC (August 1996)
    \item ``A Framework for Control of a Camera Head'', Claus S. Andersen, EE @
          AUC (March 1996)
    \item ``View Planning for Quantification of Local Geometry'', Claus Madsen,
          EE @ AUC (Oct. 1994).
  \end{enumerate}

\item[Courses taught:]
  \begin{enumerate}
  \item Topics in Robotics --- Seminar Series (CSE 290 --- 2019--)
  \item Introduction to Robotics (CSE276A --- 2018--)
  \item Mathematics for Robotics (CSE276C --- 2018--)
  \item Introduction to Robotics (CSE291D --- 2018)
  \item Mathematics for Robotics (CSE291G --- 2018)
  \item Pattern Recognition (CSE 291 --- 2017)
  \item Pattern Recognition (CS7616 --- 2016)
  \item Introduction to Robotics --- Graduate (CS 7785 --- 2015)
  \item Industrial Robotics (Professional Education Course @ NIST/DLPE
          --- 2012)
  \item Software Engineering in Robotics (CS8803 --- 2010)
  \item Multi Disciplinary Robotics Research (CS8750/8751 --- 2009 --2016)
  \item Applied Estimation for Robotics (CS8803 --- 2009, 2010, 2012, 2013)
  \item Introduction to Robotics and Perception (CS3630 --- 2008, 2009, 2015)
  \item Mobile Manipulation (CS4632B/8803 --- 2007, 2008)
  \item Freshman Leap --- Section Lead (CS1101 --- 2007, 2009)
  \item Artificial Intelligence --- An Introduction (undergraduate,
    2004--2005 @ KTH)
  \item Behavior Based Robotics (Graduate, 1998--2004 @ KTH)
  \item Autonomous Systems (undergraduate, 2002--2006 @ KTH)
  \item Urban Robotics (Industrial --- Professional Education, 2002 @ KTH)
  \item Autonomous Robots (Industrial --- Professional Education, 2000 @  KTH)
  \item Mobile Robotics (Graduate, 1996--1999 @ KTH)
  \item Computer Vision Techniques and Projective Geometry (Graduate,
    1994--1996 @ KTH)
  \item Discrete Mathematics (Graduate, 1993--1996 @ AUC)
  \item Expert Systems (Graduate, 1993--1996 @ AUC)
  \item Analysis and Design of Algorithms and Data-structures
    (Undergraduate, 1995 @ AUC)
  \item Expert System Technology (Industrial --- Professional Education,
    1994 @ AUC)
  \item Biological Vision (Graduate, 1993 @ AUC)
  \item Structured programming (Undergraduate, 1992 @ AUC)
  \item C-programming (Undergraduate, 1991 @ AUC)
  \item Motion Analysis (Graduate, 1990 @ AUC)
 \end{enumerate}

\item[Other Teaching Activities]
\begin{itemize}
  \item Chair of Committee for definition of Undergraduate AI major in Computer
        Science, UC San Diego (2023--2024)
        
  \item Chair of committee for Masters of Robotics, Engineering, 
    UC San Diego (2019--2021)

  \item Chair of Robotics Specialization --- Computer Science, UC San
    Diego (2017--)

  \item Supervised or Co-supervised 200+ M.Sc.\ level projects,
    many basic and advanced B.Sc.\ projects ($>$60) in Electronic
    Engineering, Computer Engineering and Computer Science.

  \item Chairman --- Engineering of Computer Based Systems education at
    KTH (adopted- Spring 2000).

  \item Chairman of committee for specification of new B.Sc. Electrical
    and Electronic Engineering curriculum at the Faculty of Science and
    Technology, Aalborg University.  The new curriculum was implemented
    from July 1996.

  \item Designed and implemented a B.Sc. specialization in E.E.
    entitled ``Industrial Computer Engineering'', Aalborg University, in
    1994.  The specialization was successfully implemented on a trial
    basis (June 1994--July 1996).

  \item Coordinator of E.E. Specialization in Computer Engineering (June
    1993--December 1995)
\end{itemize}
\end{cvlist}

% -------------------------------------------------------------

\noindent{\bf \large V. SCHOLARLY ACCOMPLISHMENTS}

\begin{cvlist}{Publications}
\item[Books]
  \begin{btSect}{books}
    % \section{Books}
    \btPrintAll
  \end{btSect}

\item[Book Chapters]
  \begin{btSect}{book-chapters}
    % \section{Book chapters}
    \btPrintAll
  \end{btSect}

\item[Edited Journal Issues]
  \begin{btSect}{journal-issues}
    % \section{Edited Journal Issues}
    \btPrintAll
  \end{btSect}

\item[Ref.~Jour.~Papers]
  \begin{btSect}{articles}
    \btPrintAll
  \end{btSect}

\item[Ref.~Conf.~Papers]
  \begin{btSect}{conf-papers}
    \btPrintAll
  \end{btSect}

\item[Theses]
  \begin{btSect}{theses}
    \btPrintAll
  \end{btSect}

\item[Reports]
  \begin{btSect}{reports}
    \btPrintAll
  \end{btSect}

\end{cvlist}

\begin{cvlist}{Keynote / Plenary Presentations}
  \item
  \begin{enumerate}
    \item ``AI by 2030'', Temasek Connection, Singapore, Nov 2024            
  \item ``The 2024 US National Robotics Roadmap'', US Congress, Apr 2024
  \item ``Robot Autonomy --- A perspective'', {\em Xponential 2024}, San Diego, Apr 2024
  \item ``A Perspective on AI'', {\em LGIM Economic Forum}, London, Apr 2024
  \item ``Micro-mobility'', {\em Norte Dame}, Distinguished Lecture, Feb 2024    
  \item ``A perspective on robotics'', {\em MnRobotics}, Minneapolis, Nov 2023
  \item ``A perspective on robotics'', {\em Robots and Drones}, FF Ventures, New York, Nov 2023
  \item ``A perspective on robotics'', {\em Wild Robots}, Aarhus, Aug 2023
  \item ``Sensor Fusion for Autonomous Driving'', {\em Fusion 2023},
          Charleston, SC, June 2023
  \item ``Futures of Robotics'', {\em GE Edge}\@ Conference, Sep 2022
  \item ``Mega-Trends and Robotics'', Keynote, {\em NVIDIA GTC}, Sep. 2022
  \item ``Manipulation in Clutter'', Keynote, {\em The 2022 IEEE Intl Conf.\ on
          Mechatronics and Automation}, Guilin, Guangxi, China, Aug 2022.
  \item ``Autonomous vehicles for micro-mobility in urban
    environments'',  USC Distinguished Lecture, Apr 2022
  \item ``People Centered Robotics'', {\em FIRE conference}, San Diego, Mar 2022
  \item ``Empowering People using Robots'', {\em AAAS Annual Meeting}, Philadelphia, Feb 2022
  \item ``Robotics for Good'', {\em UN/ITU Symposium on AI for Good}, Geneva, September 2021
  \item ``Robot Assembly in Clutter'', {\em IEEE Conf on Mechatronics}, Beijing, August 2021
  \item ``Challenges in Robotics'', {\em UBS Investor Forum}, August 2021
  \item ``A perspective on robotics --- Update on 2020 Roadmap'', {\em  AUVSI Annual Meeting}, July 2021
  \item ``Long-term deployment of micro-mobility systems'', {\em CVPR
          Workshop on Robot Systems for Unstructured Environments}, June  2021
  \item ``A perspective on robotics'', {\em CLSA Tokyo Economic Forum}, Tokyo, May 2021
  \item ``Robotics and Automation for US'', {\em Wells Fargo --- CEO Forum}, New York, NY May 2021
  \item ``The Future of Automation'', {\em Wellington Capital}, Singapore, April 2021
  \item ``Robotics in a Post-COVID Society'', {\em COSGUN-2020}, Seoul, November 2020
  \item ``What is next in Robotics'', {\em Robotics and Automation}, September 2020
  \item ``Addressing COVID-19'', {\em Briefing to Congressional Staff}, Washington DC, July 2020
  \item ``Impact of AI Research'', {\em Congressional Briefing}, Washington DC, December 2019
  \item ``Autonomous Driving Vehicles'', {\em AAA Summit}, San Diego, November 2019
  \item ``Robotics and AI'', {\em Collaborative Robotics, AI and Vision}, San Jose, November 2019
  \item ``A Perspective on Robotics'', General Electric Leadership Summit, New York, October 2019
  \item ``Human-Robot Collaboration'', ISRR, Hanoi, October 2019
  \item ``A Perspective on AI'', Danish Innovation Fund, September 2019
  \item ``Exploration and mapping by mixed human-robot teams'', {\em
      IEEE Intl.\ Symp.\ on Safety, Security and Rescue Robotics}, Wurzburg, Sept. 2019.
  \item  ``Advances in Robotics'', {\em A.T.\ Kearney CEO Forum}, Mallorca --- Spain, July 2019
  \item ``A perspective on Robotics'', {\em RSS Pioneer Keynote}, Freiburg, June 2018
  \item ``What to expect for 2020?'', {\em Robot Summit 2019}, Boston, June 2019
  \item ``A Perspective on consumer robotics'', {\em Consumer Technology Association}, San Francisco, May 2019
  \item ``AI for Even Better Robots'', {\em LG Keynote at CES}, Las Vegas, Jan 2019
  \item ``Robots, Fog and Clouds in a New Economy'', {\em ROS-Industrial 2018}, Stuttgart, Dec 2018
  \item ``The New Robot Economy'', {\em New York Stock Exchange}, Oct 2018
  \item ``Multi-Modal Processing for Intelligent Systems'', {\em IEEE Intl. Conf on Multi-Media},
          San Diego, CA, Jul 2018
  \item ``Semantics for Mobile Robots'', {\em IEEE Semantic Computing}, Riverside, CA, Feb 2018
  \item ``A perspective on service robotics'', {\em ISRR 2017}, Puerto Varas, Chile, Dec 2017
  \item ``Robotics in China'', {\em US China Commission}, Washington DC, March 2017 (testimony)
  \item ``Opportunities and Challenges in Robotics'', {\em MARS}, Palm Springs, March 2017
  \item ``A perspective on robotics'', {\em RoboUniverse}, San Diego,  December 2016
  \item ``A vision for robotics'', {\em CTO Forum}, Half Moon Bay, November 2016
  \item ``Metrology for the new industry'', Zeiss Forum, Detroit,  November 2016
  \item ``A perspective on manufacturing'', {\em IEEE Futures Forum},  October 2016
  \item ``The Future of Everything'', {\em US-Austria Summit},  September, 2016
  \item ``A Roadmap to the future'', {\em RoboBusiness-2016}, Odense,  June 2016
  \item ``An Overview of Collaborative Robots'', RIA Collaborative Robotics Symposium, Boston, May 2016
  \item ``A perspective on robotics'', CUNY Lectures on Design and Technology, April 2016
  \item ``Robot Opportunities with a Focus on Asia'', CLSA, Tokyo,  Dec. 2015
  \item ``Collaborative Robotics --- A Perspective'', RIA/Collaborative Robotics Workshop, Pittsburgh, Oct 2015.
  \item ``A perspective on robotics'', Carnegie, Copenhagen, Oct. 2015
  \item ``Futures of Manufacturing'', ConfigIt Summit, Georgia, Sep 2015
  \item ``2D and 3D Vision for Robotics'', {\em ICVS-2015}, Copenhagen, Jul. 2015
  \item ``Vision based robotics'', BAU Futures of Robotics, Istanbul, Jun. 2015
  \item ``Collaborative Robotics'', ISR/Automate 2015, Chicago, March 2015
  \item ``The Future of Fabrication'', IEEE {\em Time Symposium}, San Jose, October 2014
  \item ``A perspective on collaborative robotics'', {\em RIA Collaborative Robotics}, San Jose, October 2014
  \item ``Robot Dreaming'', {\em CLSA Asian Forum}, Hong Kong, Sep 2014
  \item ``The confluence of robotics and automation'', {\em CASE Keynote}, Taipei, August 2014
  \item ``Collaborative Robotics'', RIA Workshop, Boston, April 2014
  \item ``Future opportunities in Robotics'', {\em Economic Forum}, Tokyo, March 2014
  \item ``Cognitive Robotics'', Karles Invitational, {\em Navy Research Laboratory}, January 2014
  \item ``A perspective on the future of robotics'', {\em GE Leadership Conference on Robotics}, Albany, Dec 2013.
  \item ``Design of cooperative robot systems''. UT Arlington {\em Distinguished Engineering Lecture}, Dec. 2013.
  \item ``Economic driver for robotics'', {\em Robot Business}, St.\ Monica,  Oct. 2013
  \item ``Robots for Everyone'', {\em TEDxEmory}, Atlanta, April, 2013
  \item ``Examples of next generation robot systems'', 5th Annual IEEE
    International Conference on Technologies for Practical Robot
    Applications (TePRA), Boston, April 2013.
  \item ``The impact of robotics on economic growth'', {\em Automate/ProMAT}, Chicago, IL, Jan 2013.
  \item ``A perspective on robotics'', IROS --- International Research  Panel, Oct 2012.
  \item ``Robotics and autonomous cars'', {\em AUVSI Driverless cars symposium}, Detroit, June 2012
  \item ``Setting an agenda for robotics'', Dutch Government  Conference, Amsterdam, June 2012.
  \item ``A vision for robotics'', USC Futures of Robotics Symposium, Los Angeles, CA.\  Dec. 2011
    \item ``A vision for the future of robotics'', Intl.\ Symposium on Robot Systems, San Francisco, CA, Oct 2011
  \item ``A Roadmap for Robotics'', National Science Foundation,  Washington DC, June 2010.
  \item ``A Vision for US Robotics'', Booz Allen Hamilton ---  Distinguished Lecture, Washington DC, April 2010.
  \item ``Cognitive systems and a vision for the road ahead'', IRT  Symposium, Tokyo, May 2010.
  \item ``A Robotics Roadmap for the Future'', AUVSI Annual Meeting, Huntsville, AL, Mar 2010.
  \item ``A US Roadmap for Robotics'', The Netherlands Office for Science and Technology Annual Conference, The Hague, Nov 2009.
  \item ``Leonardo Da Vinci --- Machines \& Robots'', High Museum of Modern Art, Atlanta, GA, June 11, 2009
  \item ``Robotics Roadmap: Internet to Robotics'', US Congressional Caucus, Washington, DC, May 23, 2009.
  \item ``Human Augmented Mapping'', Franklin Symposium to honor Dr.\ Ruzena Bajcsy, University of Pennsylvania, Philadelphia,
    PA.\ April 2009.
  \item ``From Internet to Robotics'', Schunk Expert Days, Stuttgart, Germany, February, 2009.
  \item ``Mobile Manipulation Systems'', {\em Intl. Conf on Control
    and Automation Systems}, Seoul, Korea, October 2008.
  \item ``Evaluation of Ground Robots for Military Use'', {\em
    European Land Robot Trial (ELROB)}, Hammelburg, DE, July 2008.
  \item ``Deployment of Robots for Economic Growth'', {\em
    International Conference on Advanced Robotics}, Jeju Island,
    Korea, August 2007.
  \item ``Vision for Cognitive Systems'', {\em Scandinavian Conference
    on Image Analysis}, Aalborg, DK, June 2007.
  \item ``Industrial Applications of Robotics'', {\em RoboBusiness
    07}, Boston, MA, May 2007
  \item ``Personal Robots'', {\em HRI Pioneers}, Washington, DC, March
    2007
  \item ``Semantic Mapping'', {\em Australian Robotics Conference},
    December 2006.
  \item ``Cognitive Systems for Cognitive Assistance'', {\em
    Australian Artificial Intelligence Conference}, December 2006.
  \item ``Evaluation of Robots for Human-Robot Interaction'', {\em
    Performance Metrics for Intelligent Systems Workshop}, NIST,
    Gaithersburg, August 2006
  \item ``Robot Vision - Vision or Robotics?'', {\em British Machine
    Vision Conference}, London, UK, June 2006.
  \item ``A European Perspective on Robotics'', {\em Intl. Symposium
    on Robotics, Tokyo}, Dec. 2005.
  \item ``Personal Robotics'', {\em Artificial Intelligence and
    Synthesis of Behaviour (AISB)}, Hertsfordshire, UK, April 2005
  \item ``A Game Theoretical Approach to Information Fusion'', {\em
    Fusion-04}, Stockholm, June 2004.
  \item ``Domestic Robot Systems'', {\em Mediterranean Control
    Conference}, Lisboa, PT, 2002.
  \item ``Active Vision from Multiple Cues'', {\em Biologically
    Motivated Computer Vision}, Seoul, Korea, May 2000.
  \item ``Intelligent Robot Systems'', {\em Intl. Joint Conf. on
    Artificial Intelligence}, Stockholm, August 1999.
  \item ``Computer Vision Systems'', {\em European Conference on
    Artificial Intelligence}, Amsterdam, August 1994.
  \end{enumerate}
\end{cvlist}

\begin{cvlist}{Patents and Invention Disclosures}
\item Mobile Robot, P. Jensfelt \& H.I. Christensen, World Patent
  (EP1804149).
\item F{\"o}rfarande f{\"o}r en anordning p{\aa} hjul (Eng: Methods
  for a thing on wheels), G. Zunino \& H.I. Christensen, Swedish
  Patent (SE0200197)
\item Position Estimation Method, H.I. Christensen \& G. Zunino, World
  patent (WO03062937)
\item FoD Detection using Laser Scanning, A. Trevor \&
  H. I. Christensen, GT Invention 5850 / Prov. Patent 61/694,361
\item Verification of as Built Structures, A. Trevor \&
  H.I. Christensen, GT Invention 5851 / Prov. Patent 61/694,378
\item CAD Simplification for Visual Servoing. C. Choi \&
  H.I. Christensen, GT Invention 5162
\item Mapping with Virtual Measurements, J. G. Rogers, A. Trevor and
  H. I. Christensen, GT Invention 5160
\item  Optical Measurements of Drilled Holes, K. Hatzilias, H. Bergman,
  \& H. I. Christensen, US Patent 8,842,273
\item  A method for relieve confusion in Alzheimer patients based on
  in-Ear EEG monitoring, H. I. Christensen \& T. Anderson
  (Provisional 2016)
\item Robotic Destination Dispatch Sytem for Elevators and Methods for
  Making and Using Same, A. Cosgun \& H. I. Christensen, 2018
\end{cvlist}

% -------------------------------------------------------------

\noindent{\bf \large VI. SERVICE}

\begin{cvlist}{Professional Service}%
\item {\bf Academic Community Service}
  \begin{itemize}
    \item Chair of Advisors, Robotics Program, MBZUAI, Abu Dhabi
          \cftdotfill{\cftdotsep} 2023--2024
  \item World Robotics Summit - Advisory Board (NEDO)
    \cftdotfill{\cftdotsep} 2016--2020
  \item Data Ethics Expert Group - Danish Department of Commerce
    \cftdotfill{\cftdotsep} Jan -- October 2018
  \item UMD Maryland Robotics Center - Member of advisory baord \cftdotfill{\cftdotsep} 2019--
    \item National Research Council / National Academies -\\ Panel on
    ``Automation / IT and it impact on Employment''
    \cftdotfill{\cftdotsep} 2015--2017
  \item Founder and Coordinator of US Robotics Virtual Organization
    \cftdotfill{\cftdotsep} 2012--2016
  \item Member of College Industry Council on Material Handling\\
    Education (CICHME) w. Material Handling Industry of America
    \cftdotfill{\cftdotsep} 2012--2014
  \item Member of Board - Danish Foundation for Strategic Research -\\
    Strategic Growth Technologies \cftdotfill{\cftdotsep} 2011--2014
  \item Member of NSF CISE Advisory Board \cftdotfill{\cftdotsep} 2011-2015
  \item Member of Robotics Technology Consortium (RTC) Board \cftdotfill{\cftdotsep} 2011-2014
  \begin{sublist}
    \item Senior Technology Seat/CTO (2013-2014)
  \end{sublist}
  \item Chair of CCC road-mapping committee for formulation of \\
    a national strategy for robotics \cftdotfill{\cftdotsep}
    2008--2009
  \item Member of Advisory Board - Bio-Robotics Prog. Univ. of Utah  \cftdotfill{\cftdotsep} 2008--2013
  \item Member of Advisory Board - NSF Ctr. Quality of Life Technology,
    CMU \cftdotfill{\cftdotsep} 2007--2014
  \item Member of Scientific Advisory Board, Robotics Institute, CMU. \cftdotfill{\cftdotsep}  2004--2018
  \item IEEE RAS Distinguished Lecturer in Robotics \cftdotfill{\cftdotsep} 2004-2006
  \item Member of academic board for KTH \cftdotfill{\cftdotsep}
    2003--2007
  \item Member of the Board of Trustees the Swedish Foundation for\\
    International Cooperation in Research and Higher Education --\\
    STINT, Appointed by the Swedish Government.
    \cftdotfill{\cftdotsep} 2002--2007
  \item Served on Ph.D committees in Norway, Sweden, U.S.A., Portugal,
    France, Belgium, Canada, Australia, Netherlands, Germany Spain,
    U.K. and Denmark for a number of candidates
  \end{itemize}

\item {\bf Involvement with professional organizations}
  \begin{itemize}
  \item A3 Committee on AI \cftdotfill{\cftdotsep} 2018--
  \item Robot Industry Association (RIA) - Board Member at Large
    \cftdotfill{\cftdotsep} 2013--2016
  \item IEEE liaison with Congressional Caucus on Robotics
    \cftdotfill{\cftdotsep}  2015--2019
  \item IEEE Fellow \cftdotfill{\cftdotsep} 2015\\
    Senior Member \cftdotfill{\cftdotsep} 2008--2014\\
    Member \cftdotfill{\cftdotsep} 1988--2007\\
    Computer Society, and Robotics and Automation Society.
    \begin{itemize}
    \item IEEE Fellow Selection Committee \cftdotfill{\cftdotsep} 2017--2021
    \item RAS TAB Member at Large \cftdotfill{\cftdotsep} 2008--2009
    \item RAS Award Nominations Co-Chair \cftdotfill{\cftdotsep} 2008, 2009
    \item RAS STCP Member \cftdotfill{\cftdotsep} 2006--2009
    \end{itemize}
  \item Founding chairman for the Danish OS-9 User Group
    \cftdotfill{\cftdotsep} 1992--1992\\ Board member
    \cftdotfill{\cftdotsep} 1994--1996
  \item Danish Chapter of the International Association of
    Pattern\\ Recognition, Secretary \cftdotfill{\cftdotsep}
    1989--1994
  \item Founding chairman of the Danish Silicon Graphics \\Users Group
    \cftdotfill{\cftdotsep} 1993--1995
  \item Co-editor of UN/IFR World Robotics --- Section on
    Service\\ Robotics (w. Martin H\"agele \& Jan Karlsson)
    \cftdotfill{\cftdotsep} 2002--2003
  \end{itemize}

\item {\bf Campus Service}
  \begin{itemize}
  \item CSE MSCOM member \cftdotfill{\cftdotsep} 2017 --
  \item Robotics Area Hiring Chair \cftdotfill{\cftdotsep} 2017 --
  \item CSE Robotics Concentration Chair \cftdotfill{\cftdotsep}
    2017 --
  \item HDSI Senior Recruiting Committee  \cftdotfill{\cftdotsep} 2019--2020
  \item Robotics Area Chair (IC/CoC) \cftdotfill{\cftdotsep} 2015--2016
  \item Devices Thread Coordinator (CoC) \cftdotfill{\cftdotsep} 2015--2016
  \item Member of CoC Dean 5 year review committee
    \cftdotfill{\cftdotsep} 2015
  \item PhD Recruiting Chair, SIC \cftdotfill{\cftdotsep} 2014--2015
  \item Member of AE faculty hiring committee \cftdotfill{\cftdotsep}
    2013--2014
  \item Member of ISYE Coca-Cola Chair Search Committee
    \cftdotfill{\cftdotsep} 2013--2014
  \item Member of External Faculty Board, Georgia Tech Manufacturing\\
    Institute \cftdotfill{\cftdotsep} 2012--2014
  \item Member of Interim Steering Committee for Institute for Big
    \\Data ~ \cftdotfill{\cftdotsep} 2012--2013
  \item Chair Search Committee for Chair of School of Computer\\
    Science, CoC, \cftdotfill{\cftdotsep} 2011--2012
  \item HUSCO-Ramierez Search Committee, School of Mechanical\\
    Engineering, Spring \cftdotfill{\cftdotsep} 2011--2013
  \item Co-chair of IC Awards Committee, \cftdotfill{\cftdotsep} 2012
  \item Member of IC Awards Committee, \cftdotfill{\cftdotsep} 2011
  \item Member of CoC RPT Committee, \cftdotfill{\cftdotsep} 2009--2013
  \item Member of Selection Committee for Dean of College of
    Computing, \cftdotfill{\cftdotsep} 2009--2010
  \item Member of School Chair Evaluation Committee - Interactive
    Computing, \cftdotfill{\cftdotsep} 2009
  \item Member of Selection Committee for Senior Vice Provost for \\
    Research and Innovation (SVPRI), \cftdotfill{\cftdotsep} 2007
  \end{itemize}

\item {\bf Reviews for conferences and journals}
  \begin{itemize}
  \item Performed journal reviews for IEEE Trans.  on Patt.  Anal.
    Mach.  Intell., Pattern Recognition, Pattern Recognition Letters,
    Intl.  Jour.  of Patt.  Recog.  and Artificial Intelligence,
    Artificial Intelligence Journal, Robotics and Autonomous Systems,
    Medical and Biological Engineering, Image and Vision Computing,
    Computer Vision and Image Understanding, IEE Proceedings: Signals,
    Speech and Vision, IEEE Signal Processing, IEEE Robotics and
    Automation, Machine Vision and Applications, IJCV, and Artificial
    Intelligence.
  \item Intl. Conf. on Patt. Rec., Technical Program Committee,
    Istanbul, August, 2010.
  \item Robotics Science and Systems, Associate Editor, Zaragoza, Jun
    2010.
  \item Intl. Conf on Robotics and Automation, Associate Editor, Kobe,
    JP, May 2009
  \item BioRobotics 2006, PC member, Pisa, February 2006.
  \item International Symposium on Robot Systems (IROS), PC-member, Sendai,
    September 2004.
  \item Intl Symposium on Robotics, member of programme committee,
    Paris, June 2004.
  \item Information Fusion 2004, PC-member, Stockholm, June 2004.
  \item Intl. Conf on Robotics and Automation, member of
    prog. committee, Sendai, May 2004.
  \item Intl. Conf on Robotics and Automation, Member of Programme
    Committee, Taiwan, Sept. 2003.
  \item Multi-Sensory Fusion, MFI-2003, Member of Programme Committee,
    NINII, Tokyo, July Award29-August 1, 2003.
  \item International Conference on Advanced Robotics, Member of
    Programme Committee, Coimbra, June 2003.
    \\item Mediterranean Control and Automation Conference, Member of
    Programme Committee, Lisboa, July 2002
  \item European Workshop on Robot Learning, Member of programme
    committee, Prague, September 2001.
  \item IARP Workshop on Technical Challenge for Dependable Robots in
    Human Environments, Member of Programme Committee, Seoul, Korea, May
    2001.
  \item Scandinavian Conference on Artificial Intelligence, Member of
    PC, Odense, Denmark, February, 2001.
  \item International Conference on Robot Systems (IROS), Member of
    Programme Committee,  Tokyo, Japan, October 2000.
  \item International Conference on Pattern Recognition, Member of
    Programme Committee, Barcelona, August 2000.
  \item Intelligent Autonomous Systems -- 6, Member of
    International Advisory Board, Venice, July 2000.
  \item 6th European Conference of Computer Vision, Member of Programme
    Committee, Dublin, June 2000.
  \item International Joint Conference of Artificial Intelligence,
    Member of Programme Committee, August 1999.
  \item European Conference of Artificial Intelligence, Member of
    Programme Committee, Brighton (UK), 23--28 August 1998
  \item Empirical Evaluation of Methods in Computer Vision, IEEE
    Workshop, Member of Programme Committee, Santa Barbara, Ca, June
    1998.
  \item 5th European Conference on Computer Vision, Member of Programme
    Committee, Freiburg, June 1998.
  \item Sensory Fusion and Decentralized Control in Autonomous Robotic
    Systems, SPIE Conference 3209, Pittsburgh, PA, Member of Programme
    Committee, October 1997.
  \item 5th International Robotics Symposium (IROS), Member of Programme
    Committee, Japan, August 1996.
  \item 14th Intl Conf on Pattern Recognition, Member of Program
    Committee, Vienna, August 1996.
  \item 4th Symposium on Intelligent Robotics Systems, member of
    programme committee, Lisbon, Portugal, July 1996.
  \item 4th European Conference on Computer Vision, Member of Programme
    Committee, Cambridge, May 1996.
  \item IEEE Workshop on Computer Vision. Member of Program Committee,
    Miami, December 1995.
  \item Symposium on Intelligent Robotics Systems '95: Member of
    Programme Committee, Pisa, July 1995.
  \item 9th Scandinavian Conference on Image Analysis: Danish Member of
    Program Committee, Uppsala, May 1995.
  \item IEEE Applications of Computer Vision, Member of program
    committee, Sarasota, Fl., December 1994.
  \item Intelligent Robotics Systems '94: Member of Program Committee,
    Grenoble, July 1994.
  \item Intelligent Robotic Systems '93: Member of Program committee,
    Zakopane, July 1993.
  \item SPIE Application of AI XI:\ Machine Vision and Robotics, Member
    of Programme Committee and Session Chair. Orlando, April 1993.
\end{itemize}

\item {\bf Professional reviews}
  \begin{itemize}
  \item Review of Graduate Programs, Faculty of Engr., Aalborg University, Aug 2018.
  \item Project and lecture reviewer at Faculty of Engr., University of Trondheim, Norway, 1994--1997.
  \item Served on professional appointment committees in Denmark,
    Spain, Sweden, Norway, United Kingdom, Belgium, Italy, France,
    Germany, Switzerland, South Korea, UAE, and U.S.A.
  \end{itemize}

\item {\bf Reviews for funding agencies}
  \begin{itemize}
  \item European Commission - Center Grants \cftdotfill{\cftdotsep} Sep.~2023
  \item DFG - SFB Panel in Robotics / Embedded Systems,
    Bonn,\cftdotfill{\cftdotsep} Jun. 2018, Oct.~2023
  \item EPSRC UK Robotics and Artificial Intelligence Hubs in Extreme
    and Challenging Environments \cftdotfill{\cftdotsep} Jun. 2017
  \item EPSRC UK Robotics Research  \cftdotfill{\cftdotsep} Jan. 2017
  \item Italian Institute of Technology (RPT) \cftdotfill{\cftdotsep}
    Jun. 2015
 \item JST - ERATO Selection Committee \cftdotfill{\cftdotsep} Jan. 2014
 \item EU Cognitive Systems, IP Reviewer, \cftdotfill{\cftdotsep}, 2013
 \item NSF Expedition Panel (phase II), \cftdotfill{\cftdotsep},
   Spring 2012
 \item SSF Successful Research Leaders - Sweden,
   \cftdotfill{\cftdotsep} Spring 2011
 \item US Army Basic Research Review Panel, \cftdotfill{\cftdotsep} July 2010
 \item SSF Future Research Leaders - Sweden, \cftdotfill{\cftdotsep} June 2010, June 2013
 \item High Technology Foundation - Denmark, \cftdotfill{\cftdotsep}  May 2010
 \item NSF ad-hoc project reviews \cftdotfill{\cftdotsep}, 2009--
 \item NSF IGERT Panel, \cftdotfill{\cftdotsep} May/June 2009.
 \item NSF Computer Infrastructure Grant Panel,  \cftdotfill{\cftdotsep} Nov. 2008
 \item Army Research Laboratory, Basic Research Review, \cftdotfill{\cftdotsep}  Spring 2008
 \item Adviser/Expert to EU DG-III Long Terms Research Office in the\\
   areas of ``robotics'' and ``computational vision''
   \cftdotfill{\cftdotsep}  1995--2006
 \item Scientific Advisor to the Swedish Foundation for International\\
   Cooperation in Research and Higher Education,  STINT
   \cftdotfill{\cftdotsep}  2000--2002
 \item Member of the Danish Reviewer Panel for Computer Science,\\
   Ministry of Education, \cftdotfill{\cftdotsep}  1995--1998, 2003--2005.
 \item Proposal reviewer for ESPRIT DG-XIII Basic Research office for\\
   the ESPRIT III and IV call for proposals and DG-III Long Term\\
   Research Office \cftdotfill{\cftdotsep}  1999--2004.
 \item Member of SSF Japan committee for Swedish - Japanese\\
   Collaboration on Interdisciplinary Research \cftdotfill{\cftdotsep}
   2001--2002
 \item Member of the Danish Reviewer Panel for Electronic Engr.,\\
   Ministry of Education, \cftdotfill{\cftdotsep}  1995--1998.
 \item Action reviewer for ESPRIT DG-III Basic Research, Long Term
   Research,\\ and Essential Technologies offices for several EU
   Projects. \cftdotfill{\cftdotsep} 1993--1998
 \item Member of International Review Panel: ``Embedded Systems'' for\\
   the Swedish National Board for Industrial and Technical Development,\\
   NUTEK, \cftdotfill{\cftdotsep} May 1996.
 \end{itemize}
\end{cvlist}

\begin{cvlist} {\bf Editorial Leadership} 
  \item Associate Editor {\em Field Robotics}, \cftdotfill{\cftdotsep} 2019--
  \item Associate Editor for {\em Science Robotics}, \cftdotfill{\cftdotsep} 2017--
  \item Editorial Board of {\em Advanced in Interaction
    Studies},\cftdotfill{\cftdotsep} 2010--
  \item Associate Editor of MIT Press series on ``Intelligent Robotics
    and Autonomous\\ Agents'',\cftdotfill{\cftdotsep} 1997--
  \item Advisory Board Member {\em AI Magazine} \cftdotfill{\cftdotsep} 2021--2023
  \item Chair Advisor board for {\em IEEE Transactions on Automation
    Science\\ and Engineering (T-ASE)}  \cftdotfill{\cftdotsep} 2014--2015
  \item Co-editor-in-chief for {\em Briefs in AI}, Atlantic Press,
    \cftdotfill{\cftdotsep}  2015--2018
  \item Associate Editor {\em Journal of Field Robotics}
    \cftdotfill{\cftdotsep} 2014--2020
  \item Advisor board for {\em IEEE Transactions on Automation
    Science\\ and Engineering (T-ASE)} \cftdotfill{\cftdotsep} 2013--2014
  \item Senior Advisory Board {\em Journal of Human-Robot Interaction}
    \cftdotfill{\cftdotsep} 2012--2016
  \item Co-Editor in Chief of {\em Trends and Foundations in
    Robotics}, \cftdotfill{\cftdotsep} 2009--2020
  \item Associate Editor of {\em Image and Vision
    Computing},\cftdotfill{\cftdotsep} 2009--2015
  \item Associate Editor of {\em Service Robotics},
    \cftdotfill{\cftdotsep} 2005--2010
  \item Associate Editor of {\em Autonomous Robots},
    \cftdotfill{\cftdotsep} 2005--2009
  \item Associate Editor of {\em International Journal of Robotics
    Research},\cftdotfill{\cftdotsep} 2002--2021
  \item Associate Editor of ``Springer Tracts in Advanced
    Robotics'',\cftdotfill{\cftdotsep} 2001--2015
  \item Associate Editor of AAAI {\em AI
    Magazine}\cftdotfill{\cftdotsep} 2000--2007
  \item Associate Editor of {\em Journal of Pattern Recognition and
    Artificial Intelligence},\\ WSP \cftdotfill{\cftdotsep} 1997--2005
  \item Associate Editor of {\em Journal of Machine Vision and
    Applications},\\ Springer Verlag \cftdotfill{\cftdotsep} 1996--2004
  \item Associate Editor of IEEE {\em Transactions on Pattern Analysis
    and Machine\\ Intelligence} \cftdotfill{\cftdotsep} 1999--2003
  \item Associate Editor {\em Robotics and Autonomous Systems}
    journal, Elsevier, \\Competition Corner, \cftdotfill{\cftdotsep}
    1999--2002
\end{cvlist}

\begin{cvlist}{Leadership in organisation of meetings:}
  \item ICRA-2028, General co-chair, Guadaljara, MX \cftdotfill{\cftdotsep} May 2028
  \item IROS-2026, Finance co-chair, Pittsburgh, PA \cftdotfill{\cftdotsep} October 2026
  \item ISER-2025, Co-Organizer, Santa Fe, NM, \cftdotfill{\cftdotsep} October
  2025
  \item ISRR-2024, Co-organizer, Los Angeles, CA \cftdotfill{\cftdotsep}
  November 2024
  \item EON - Edge of Now, Advisor, Laguna Beach, \cftdotfill{\cftdotsep}
  September 2023
  \item Intl Conference on Vision Systems - PC-Co-Chair, Vienne,
  \cftdotfill{\cftdotsep} Oct 2023
  \item EON - Edge of Now, Chief Advisor, Palm Springs, \cftdotfill{\cftdotsep}
  October 2022
  \item Gordon Research Conference on Robotics, General Chair, Ventura,
  \cftdotfill{\cftdotsep}, August 2022
  \item ICPR-2022 - General Co-Chair, Montreal, \cftdotfill{\cftdotsep} August
  2022
  \item ICRA-2022 - Chair of Forums, Philadelphia, \cftdotfill{\cftdotsep} May
  2022
  \item IROS Corporate Relations, Prague \cftdotfill{\cftdotsep} Sep 2021
  \item Intl Conference on Vision Systems - PC-Co-Chair, Vienne,
  \cftdotfill{\cftdotsep} Sep 2021
  \item ISER-2021 - Organizing Committee, Malta \cftdotfill{\cftdotsep} Mar 2021
  \item IROS-2020 - Corporate Relations, \cftdotfill{\cftdotsep} Oct 2020
  \item ISRR-2019 - Co-organizer, Hanoi \cftdotfill{\cftdotsep} Oct 2019
  \item ICRA-2018 - Government Forum Co-Chair \cftdotfill{\cftdotsep} June 2018
  \item ICRA-2017 - US Program Co-Chair, Singapore \cftdotfill{\cftdotsep} May
  2017
  \item ISRR-2015 - US Chair, Italy \cftdotfill{\cftdotsep} Sep. 2015
  \item 5th IDEAS - Surgical Robotics WS, Harvard Medical School, Co-organizer,
  \cftdotfill{\cftdotsep} Mar. 2015
  \item IAS-13, Steering Committee, Venice, \cftdotfill{\cftdotsep} Jul. 2014
  \item 4th IDEAS - Surgical Robotics WS, Harvard Medical School, Co-organizer,
  \cftdotfill{\cftdotsep} Apr. 2014
  \item 8th Schunk Expert Days, Co-Chair, Lauffen, DE, \cftdotfill{\cftdotsep}
  Feb 2014
  \item ISRR-2013, Finance/US Chair, Singapore, \cftdotfill{\cftdotsep} Dec.
  2013
  \item 3rd IDEAS - Surgical Robotics WS, Harvard Medical School, Co-organizer,
  \cftdotfill{\cftdotsep} Apr. 2013
  \item 7th Schunk Expert Days, Co-Chair, Lauffen, DE, \cftdotfill{\cftdotsep}
  Feb 2013
  \item IROS-2012, Senior Program Committee, Portugal,\cftdotfill{\cftdotsep}
  Oct. 2012
  \item IAS-12, Regional Program Chair, Seoul, Korea, \cftdotfill{\cftdotsep}
  Aug. 2012
  \item ICRA-2012, General Co-Chair, \cftdotfill{\cftdotsep} May 2012
  \item 2nd IDEAS Symposium, Co-organizer, Harvard Medical School,
  \cftdotfill{\cftdotsep} April 2012
  \item IROS-2011, Special Symposium Committee, \cftdotfill{\cftdotsep} Oct.
  2011
  \item ISRR-2011, General Chair, Flagstaff, AZ, \cftdotfill{\cftdotsep} Aug.
  2011
  \item Ro-Man 2011, General Chair, Atlanta, GA. \cftdotfill{\cftdotsep} Aug.
  2011
  \item Schunk Expert Days, Co-Chair, Lauffen, DE, \cftdotfill{\cftdotsep} Feb
  2011
  \item IAS-11, Program Chair, Ottawa, Canada, \cftdotfill{\cftdotsep} Aug. 2010
  \item Schunk Expert Days, Co-Chair, Lauffen, DE, \cftdotfill{\cftdotsep} Feb
  2010
  \item Ro-Man 2009, PC - Co-Chair (Americas), Toyama - JP,
  \cftdotfill{\cftdotsep} Sep. 2009.
  \item IROS-2009, US Program Chair, Lucerne, CH, \cftdotfill{\cftdotsep} Sep.
  2009.
  \item Schunk Expert Days, Co-Chair, Lauffen, DE, \cftdotfill{\cftdotsep} Feb
  2009
  \item HRI - 2008, Senior Programme Committee, Amsterdam,
  \cftdotfill{\cftdotsep} April 2008,
  \item IROS, US Program Chair, Nagasaki, \cftdotfill{\cftdotsep} Nov. 2007.
  \item ICRA-2007, Programme Co-chair -- Europe, Rome, \cftdotfill{\cftdotsep}
  April 2007.
  \item HRI 2007, Senior Program Committee, Washington, \cftdotfill{\cftdotsep}
  March 2007.
  \item HRI 2006, Senior Programme Committee, Utah, \cftdotfill{\cftdotsep}
  March 2006,
  \item 19th Intl Joint Conf on Artificial Intelligence, Edinburgh,\\ Senior
  Program Committee, \cftdotfill{\cftdotsep} Aug. 2005.
  \item ICRA-2005, European Programme Co-Chair, Spain, \cftdotfill{\cftdotsep}
  April 2005.
  \item Field deployable robots, NATO IST Workshop, Co-chair, Bonn,
  \cftdotfill{\cftdotsep} Sept 2004
  \item RAS-IFRR Summer school on ``Human-Robot Interaction'', Co-organiser,
  \\Volterra, \cftdotfill{\cftdotsep} July 2004.
  \item Robotics demining, Brussels, co-organiser, Belgium,
  \cftdotfill{\cftdotsep} June 2004.
  \item Wallenberg Symp. on Sensing and Feeling, Co-organiser,
  \cftdotfill{\cftdotsep} May 2004.
  \item Educational Robotics, Co-organiser, ICRA-04 Workshop, New Orleans,
  \cftdotfill{\cftdotsep} April 2004.
  \item Challenges in Cognitive Vision, NIPS workshop, Co-organiser,
  \cftdotfill{\cftdotsep} Dec. 2003
  \item Cognitive Vision Systems, Dagstuhl Seminar, Co-organiser,
  \cftdotfill{\cftdotsep} Oct. 2003
  \item Nobel Symposium on Neural Control of Skilled Hand Movements:\\ Cognitive
  and Computational Aspects, Stockholm, Co-organiser, \cftdotfill{\cftdotsep}
  June 2003.
  \item Intl Conference on Vision Systems, Steering Committee, Graz,
  \cftdotfill{\cftdotsep} Mar. 2003.
  \item WS on Control Problems in Robotics and Automation, \\General Chair, Las
  Vegas, \cftdotfill{\cftdotsep}Dec. 2002.
  \item IROS-2002, European/African Programme Chair, Lausanne,
  \cftdotfill{\cftdotsep} Oct. 2002.
  \item International Symposium on Robotics, Service Robotics Chair, Stockholm,
  \cftdotfill{\cftdotsep} Oct. 2002.
  \item WS on Robot Dependability, IARP, Organising Committee,
  \cftdotfill{\cftdotsep} Oct. 2002.
  \item Wallenberg Symp. on Learning and Memory: Brains to Robots,\\
  Member of Organisation Committee, Stanford, \cftdotfill{\cftdotsep} Oct. 2002.
  \item Summerschool on ``Simultaneous Localisation and Mapping'', \\Organiser,
  Stockholm, \cftdotfill{\cftdotsep}Aug. 2002.
  \item ECAI, Vision-Robotics Chair, Toulouse, \cftdotfill{\cftdotsep} Aug.
  2002.
  \item ICPR, Computer Vision Co-Chair,\\ Quebec City, \cftdotfill{\cftdotsep}
  Aug 2002.
  \item Tutorial on ``Mobile Robot Programming Paradigms'', Co-organiser\\ (with
  Greg Hager, JHU), ICRA-2002, Washington, \cftdotfill{\cftdotsep}May 2002.
  \item 3rd Ws. on Empirical Eval. Methods in Comp. Vis., Co-Chair, Maui,
  \cftdotfill{\cftdotsep} Dec. 2001.
  \item WS on Computer Vision Systems, Co-Chair, Victoria,
  Canada,\cftdotfill{\cftdotsep} July 2001
  \item Modeling of Sensor Based Intelligent Robot Systems,\\
  Co-organizer, Dagstuhl, Wadern, \cftdotfill{\cftdotsep} Oct. 2000.
  \item 1st Swedish Autonomous Robotics Symposium, Co-chair, \"Orebro,
  \cftdotfill{\cftdotsep}Oct. 2000.
  \item ECAI, Area Chair (Robotics and Vision), Berlin, \cftdotfill{\cftdotsep}
  Aug. 2000.
  \item 2nd International WS on Perf Char., European Programme Chair,
  Dublin,\cftdotfill{\cftdotsep} June 2000.
  \item 1st ICVS, Las Palmas, Programme Chair, \cftdotfill{\cftdotsep} Jan.
  1999.
  \item Environmental Modelling for Mobile Robotics, Schloss Dagstuhl
  Workshop,\\ Weidern, Co-organiser, \cftdotfill{\cftdotsep} Sep. 1998.
  \item Knowledge Based Methods for Computer Vision, Schloss Dagstuhl Workshop,
  Weidern,\\ Co-organiser, \cftdotfill{\cftdotsep} Dec. 1997.
  \item 5th SIRS, Programme Chair, Stockholm, \cftdotfill{\cftdotsep} July 1997.
  \item Performance Characteristics of Vision Algorithms, Co-chair programme
  committee\\ with Prof. W. F\"orstner, Cambridge, UK, \cftdotfill{\cftdotsep}
  April 1996.
  \item Active Vision Hardware Workshop, Co-Organiser w. Prof. J.L. Crowley,
  \\Grenoble, France,\cftdotfill{\cftdotsep} Feb. 1995.
  \item Nordic Summer School on Active Vision and Geometric Modelling, \\
  Organiser. Rebild Bakker, Aalborg, \cftdotfill{\cftdotsep} Aug. 1992.
  \item SPIE Applications of Artificial Intelligence X: Machine Vision and
  Robotics, \\ Programme Committee, Organiser and chairman of session on\\ ``How
  to Design a Robot Head''. Orlando, \cftdotfill{\cftdotsep} April 1992.
  \item 7th Scandinavian Conference on Image Analysis. \\Chair of Local
  Arrangements. Aalborg, \cftdotfill{\cftdotsep} Aug. 13--16, 1991.
  \item Topical Workshop on Symbolic Reasoning in Scene Interpretation,
  Co-organiser, \\LIFIA, France. ESPRIT Vision Workshop Week. Crete,
  \cftdotfill{\cftdotsep} Sep. 1990.
  \item Topical Workshop on Perceptual Control, Co-organiser, Aalborg
  University. \\ ESPRIT Vision Workshop Week. Crete, \cftdotfill{\cftdotsep}
  Sep. 1990.
  \item 4th Aalborg Symposium on Vision: Concurrent Computer Vision '89,
  Co-organiser, \\ Institute of Electronic Systems, Aalborg,
  \cftdotfill{\cftdotsep} Jan. 24--26, 1989.
  \item 3rd Aalborg Symposium on Vision: Hybrid Methods '87, Co-organiser, \\
  Institute of Electronic Systems, Aalborg, \cftdotfill{\cftdotsep} Dec. 10--11,
  1987.
  \item 2nd Aalborg Symposium on Vision: Robot Vision '86, Co-organiser, \\
  Institute of Electronic Systems, Aalborg, \cftdotfill{\cftdotsep} Dec. 15--17,
  1986.
\end{cvlist}

\begin{cvlist}{Research Grants:}
  \item IHI:\ Evaluation of scaleable navigation framework for the real-time
  multi-robot coordination, (\$500k/yr, 2023--2026), PI
  \item Nissan: Mapping, Estimation and Planning for Intelligent Vehicles
  (\$160k/yr, 2023--2025), PI
  \item ONR:\ Naval Innovation and Translation (\$12.5M, 2023--2027), Co-PI
  \item Qualcomm: Behavior Prediction for Autonomous Vehicles (\$80k, 2023--2024), PI
  multi-human collision avoidance in the logistics field (\$1M, 2023--2025), PI
  \item ARL:\ Distributed Collaborative Intelligent Science and Technology (\$2.9M
  (UCSD), 2022--2027), Co-PI
  \item Qualcomm: Intent Recognition of Unprotected Road Users (\$80k, 2022--2023),
  PI
  \item Qualcomm: Behavior Based Path-Planning (\$80k, 2022--2023), PI
  \item ONR:\ DURIP --- Computing Resources for Machine Learning (\$575k, 2022--2023),
  Co-PI
  \item Amazon: Pedestrian Dataset (\$248k, 2022--2023), PI
\item Qualcomm: Open Source Robotics Platform (\$320k, 2021--2022), PI
\item NSF:\ AI Institute for Learning-Enabled Optimization at Scale
  (TILOS) (\$20M, 2021--2027), Co-PI
\item NSF:\ Human-centered, integrated mobility for disadvantaged
  communities in the San Diego region (\$50k, 2021), Co-PI
\item Qualcomm: Innovation Fellowship (\$100k, 2020--2021), PI
\item ONR:\ DURIP --- Instrumentation for the AeroDrome (\$500k, 2020) Co-PI
\item LGe:\ ROS 2.0 Service Robot (\$150k, 2019), PI
\item RPD Innovation: Robotics and Artificial Intelligence (AI) System for 
  Industry 4.0, (\$574k, 2019--2020), PI 
\item SPAWAR:\ Intelligent Diagnostics of V-22 Osprey (\$1M, 2020), Co-PI
\item SPAWAR:\ Data Science Support for SPAWAR 4.0 Logistics (\$500k,
  2018--2020), PI
\item TuSimple --- Assessment of Level 4 Autonomy in Trucks,  (\$250k,
  2018--2019), PI
\item ARL:\ Autonomous Resilient Cognitive Heterogeneous Swarms
  (DCIST) --- (\$6.3M, 2018--2022), UCSD PI
\item LGe:\ Tools for embedded robots development (\$125k, 2017), PI
\item Qualcomm: Long-Term Autonomy (\$80k, 2017), PI
\item NSF:\ NRI:Workers, Firms, and Industries in Robotic Regions
  (\$784k, 2016--2017), Co-PI
\item NSF:\ Revision of the national robotics roadmap (\$35k, 2016), PI
\item Boeing: Fixture Less Machining (\$200k, 2016--2017), PI
\item Boeing: Assembly Inspection using Augmented Reality (\$49k,
  2015), PI
\item Thyssen Krupp: Robot Control for Elevators, (\$80k, 2014--2015), PI
\item Boeing: Augmented Reality (\$49k, 2014)
\item NSF: NRI: Representing and Anticipating Actions in Human-Robot
  Collaborative Assembly Tasks (\$800k, 2014-2016), Co-PI.
\item NSF: NRI PI Meeting (\$114k, 2014), PI
\item NSF: EAGER - Physical Flow and other Industrial Challenges,
  (\$300k, 2014-2015), Co-PI
\item NSF: Opportunities in Manufacturing, Robotics and Computer Science
  (\$48k, 2013-2014), PI
\item Boeing: High precision robot manufacturing (\$600k, 2013-2015), PI
\item Boeing: Vision for Augmented Reality (\$17k, 2013), PI
\item NSF: NRI PI Meeting (\$104k, 2013), PI
\item Micro Autonomous Systems Technology - Army Research Laboratory
  CTA: Autonomy (2013-2017) - PI for GT (\$1.9M). Lead UPENN (Total
  \$20M) % $2M (GT), 10m in total
\item PSA: Robotics for Automotive Assembly (\$910k, 2013-2015), PI
\item BMW; Support on the Shop-Floor Using Modern Robots (\$750k,
  2012-2015), Co-PI
\item Mitsui/Motoman: Laboratory Automation (\$164k, 2012-2014), PI
\item NIST: Robots for Kitting (\$99k. 2012-2014). PI
\item Boeing: Wing Assembly (\$1.4M, 2011-2014), PI
\item NSF: Robotics Virtual Organization (\$100k, 2011-2013), PI
\item NSF: Motion Grammer Laboratory - Equipment Grant (\$330k, 2011-2013), Co-PI
\item Boeing: UGV Navigation for OmniMove (\$146k, 2010), PI
\item MSR: Computer Vision Library for RDS (\$25k, 2010), PI
\item MSR: Software Engineering for Robotics (\$75k, 2010), PI
\item NIST: Mixed Palletizing (\$170k, 2009-2012), PI
\item KOTEF: Cognitive Consumer Robots (\$2.4M, 2009-2011), GT-PI
\item NSF: Young Researchers Workshop - 2009 (\$24k), PI
\item NRL: Disruptive Technologies for General Infra-structure  (\$25k, 2008-2009), PI
\item GM: Factory CoWorker - (\$400k/yr, 2008-2010), Co-PI
\item Boeing: Factory of the Future - Robotics (\$700k, 2008-2014), PI
\item KUKA: UGV Survey, KUKA Roboter, Germany, Oct. 2008 (\$6k)  PI% 18k$
\item Micro Autonomous Systems Technology - Army Research Laboratory
  CTA: Autonomy (2008-2013) - Co-PI for GT (\$4M). Lead UPENN (Total \$33M) % $2M (GT), 10m in total
\item CCC: From Internet to Robotics: The Next Transformational
  Technology (\$200k, 2008-2009), PI % $200k
\item KUKA: Unlayering, KUKA Roboter, Germany, Jul. 2008 (\$18k) % 18k$
\item KUKA: Exploratory research on diagnostics and navigation, KUKA
  Roboter, Germany, Spring 2008 (\$55k)% 55k$
\item CEC: CoSy - Cognitive Systems for Cognitive Assistants, IP
  Project, Coordinator (11M EUR, 2004-2008)
\item VR: Multi-Modal Mapping (1.7M SEK, 2006-2008), PI.% 1.7 MSEK
\item CEC: Neurobotics -- Neuroscience/Robotics, IP Project,
  Co-investigator (6.4 M EUR, 2004-2007)% 6.4 MEURO
\item CEC: Cognitive Companion -- Cogniron, IP Project,
  Co-investigator (6.8M EUR, 2004-2007) % 6.8 MEURO
\item CEC: EURON-II -- EU Network of excellence within ``beyond
  robotics''  -- Coordinator (5.8M EUR, 2004-2007) % 5.8 MEuro
\item SSF, Autonomous Systems, Principal Investigator -- Director  (7
  M SEK, 2003--2006) % 20 MSEK
\item FMV: Intelligent Unmanned Vehicles, Technology Demonstrator,
  Coordinator, (8M SEK, 2001-6). % 8000 KSEK
\item FMV: UGV control using the universal control station, 2005 (200k
  SEK) %200 kSEK
\item CEC: Cognitive AI Enabled Computer Vision network, Research
  Co-ordinator, (3.4 M EUR, 2002-2005). % 3.4 MEURO
\item STINT: Institutional Grant for KTH-ANU Collaboration in the area
  of Collaborative Robotics, Co-chair (2M SEK, 2001-2005).% 2000 KSEK
\item CEC: CogVis -- ``Cognitive Vision'', IST Research Project
  (IST-2000-29375), Coordinator, (4 M EUR, 2001--2004).% 3950 KEuro
\item FOI: ``Information Fusion'', academic co-chair,  (1.8M SEK, 2001-2004).% 1800 KSEK
\item CEC: OROCOS -- ``Open Robot Controller Software'',
  (IST-2000-31064), Co-investigator (60k EUR, 2001-2003).% 60 kEuro
\item CEC: EURON -- European Robotics Research Network
  (IST-2000-26048), Coordinator, (1.035M EUR, 2000-2003)% 1035 KEuro
\item CEC: PCCV -- ``Performance Characterisation of Computer
  Vision'',  co-principal investigator, (180k EUR, 2000-2003)% 180 kECU
\item NUTEK Complex Technical Systems ``Sensory Fusion for Robot
  Navigation'', (1.8M SEK, 1999-2001)% 1.8 MSEK
\item NUTEK Complex Technical Systems ``Architectures for Mobile
  Robotics'', (2.2M SEK, 1999--2001)% 2.2 MSEK
\item Foundation for Strategic Research: Centre for Autonomous Systems
  (63M SEK, 1997-2001). Scientific Director.%  73 MSEK
\item CEC: TMR network ``CAMERA'', co-principal investigator  (104k
  EUR, 1998-2001)% 104 kECU
\item NUTEK Exploratory Grant ``Intelligent Crane Control'',
  co-principal investigator, (450 k SEK, 1999-2000)% 450 KSEK
\item NUTEK Exploratory Programme Grant ``Intelligent Outdoor
  Vehicles'', (200k SEK, 1999)% 200 kSEK
\item STINT Visiting Professor Grant (for Prof.\ Ronald Arkin)  (755k
  SEK, 1997--1998).% 755 kSEK
\item CEC: TMR Network on ``Vision for Robot Guidance'', Co-proposer
  and local manager (1.4M EUR, 1996--1997).% 120 kECU / Total 1.4 MECU
\item CEC: TMR Network on ``Sensory Mobile Autonomous Robot Technology
  II'', Co-proposer and local manager.  (2.3M EUR, 1996--1997).% 200
  % kECU / TOTAL 2.3 MECU
\item Danish Technical Research Council: ``Reconstruction and
  Visualisation of 3D Structures based on In-vivo Image Analysis'',
  Principal Investigator (1.014M DKK, 1995--1998).% 1014 KDKK
\item EEC: European Network of Excellence in Computer Vision,
  Co-proposer and Principal Investigator (2M EUR, 1994--1998).% 2MECU
\item EEC: HCM Network on ``RETINA: Active Vision'', Co-proposer and
  local manager.  (3M EUR, 1994--1997).% 170 kECU-AUC / Total 3MECU
\item LUKAS/SPIN: Software Process Improvement Network. Funded by EU
  and Regional Council for North Jutland. Member of Board and
  Principal Investigator (8M DKK, 1995--1996).% 8 MDKK
\item EEC: HCM Network on "Sensory Mobile Autonomous Robot
  Technology", Co-proposer and local manager.  (2.9M EUR,
  1993--1995).% 210 kECU / Total 2.9 MECU
\item DOAP: Data Acquisition, - Analysis, and Presentation, LUKAS --
  Regional Development Fund, North Jutland Regional Council,
  Co-Proposer and Technical Coordinator. (6M DKK, 1993--1995). Member
  of  Executive Board for the LUKAS software quality assurance project
  (1994--1995).% 6 MDKK
\item EEC: Vision as Process-II, P-7108-VAP-II, Co-proposer and local
  manager (3M EUR, 1992--1995).% 3MECU
\item NorFa: Nordic Research Network on Computer Vision, Co-proposer
  and ass. coordinator.  (150k DKK, 1992--1995)% 150 KDKK
\item EEC: Vision as Process, BR-3038-VAP, Co-proposer and local
  manager (4.7M EUR, 1989--1992).% 4,7 MECU
\item NorFa: Nordic Ph.D Summer School, Rebild, August 1992, Proposer
  and coordinator. (200 k DKK)% 200 KDKK
% \end{enumerate}
\end{cvlist}

\begin{cvlist}{Gifts:}
  \item Northrup Grumman: Human-AI teaming (\$100k, 2024--2025)
  \item IHI: Warehouse Logistics (\$150k, 2024--2025)
  \item Mercedes-Benz: 4D-NERF study (\$50k, 2024)  
  \item ILA: Robot Manipulator Testbed (\$300k, 2023)
\item Nissan: Scene Modeling \& Planning (\$50k, 2023)
\item Northrup Grumman: Autonomous Systems (\$320k, 2021--2023)
\item Qualcomm: Robotics using the Qualcomm Platform (\$350k, 2021--2022)
\item LGe: Next Generation Home Robots (\$120k, 2019--2020)
\item Qualcomm/UCSD: Digital Collaboratory in Smart Transportation
  (\$350k, 2017--2018)
\item Honda, Collaborative Robotics (\$50k, 2018, 2019)
\item Northrup Grumman Group, AUS systems (\$100k, 2018)
\item Qualcomm: Contextual Robotics (\$800k, 2017--2018)
\item Qualcomm: Long-term autonomy (\$125k, 2017)
\item Kelly Family: \$20k grant for distinguished lecture series on
  robotics and employment (2015--2016)
\item Intel: \$10k equipment donation (2015)
\item National Instruments: \$80k equipment donation (2011--2012).
\item Coca Cola Bottling Company: \$2M equipment donation for setup of
  a logistics laboratory (2011).
\item Private Donation: \$5k award money for the Dick Volz PhD Award
  (2011)
\item General Motors: \$12k for promotion of next generation
  manufacturing (2010)
\item KUKA: Endowment (\$1.5m, 2016)
\end{cvlist}

\begin{cvlist}{Honors and Awards:}
\item ICRA-2024 - Best Paper Award - Open X-embodiment: Robotic Learning Datasets and RT-X, May 2024
\item IAS-16 Best Paper Award - TridentNet w. D. Paz, H. Zhang, June 2021
\item Silicon Valley Robotics - Community Champion Award, 2020
\item Elected Fellow, Institute of Electrical and Electronic Engineers (2015)
\item Honorary Doctorate in Engineering (Dr. Techn. h.c.), Aalborg University (2014)
\item Elected Fellow, American Association for the Advancement of Science (2013)
\item Boeing Supplier of the Year Award (2012)
\item Dean's Award, College of Computing, GT (2012)
\item The Joseph Engelberger Award, Robot Industry Associations (RIA - 2011)
\item Outstanding Innovation in Research, Faculty Award, Georgia Tech (2011)
\item Named IEEE Senior Technical Expert (2009--2012)
\item Peter Freeman Award, College of Computing, GT (2008)
\item Elected Senior Member of IEEE (2008--2014)
\item Elected Officer of International Foundation of Robotics Research
  (2003--)
  \begin{sublist}
  \item There are at anytime only 24 officers -- 8 from US, Asia and Europe,
    respectively.
  \item Treasurer/Secretary for Technical Activities - Executive
    Officer (2010--)
  \end{sublist}
\item ICRA-2004 Short-listed for best vision paper ``Measurement
  Errors in Visual Servoing'' authored by V. Kyrki, D. Kragic and H.
  I. Christensen.
\item ICRA-2003 Best Paper on Manipulation:  ``Automatic
  Grasp Planing using Shape Primitives'' authored by A. Miller and
  S. Knopp, H.I. Christensen and P. Allen.
\item IROS-2002 Best Paper Award ``Behavior Coordination for
  Navigation in Real-World Office Environments'' authored by
  P. Althaus and H.I. Christensen.
\item Jury Member of Robot Hall of Fame (2003--2013), Carnegie Mellon
  University, Pittsburgh, PA.
\item The Foundation Vision North 1991 Research Award.
  \begin{sublist}
  \item Contribution to advancement of research at the Laboratory of
    Image Analysis, Aalborg University.
  \item August 1991.
  \end{sublist}
\end{cvlist}
\end{cv}
\end{document}


%%% Local Variables:
%%% mode: latex
%%% TeX-master: t
%%% TeX-engine: default
%%% End:
